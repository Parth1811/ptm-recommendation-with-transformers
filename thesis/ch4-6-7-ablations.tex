\subsection{Ablation Studies}
\label{subsec:ablation-studies}

While the current evaluation demonstrates Cross-Select's effectiveness across multiple generalization scenarios, comprehensive ablation studies remain as important future work to validate architectural choices and explore alternative design decisions. Two ablation directions are particularly critical for strengthening the system's empirical foundation and identifying potential improvements.

First, a systematic ablation study employing models with progressively refined architectures—known a priori to exhibit superior performance on specific tasks—would validate that Cross-Select preserves canonical ordering relationships. By constructing evaluation sets where ground-truth model quality is established through extensive benchmarking (e.g., EfficientNet-B0 through B7 variants, or ResNet-18 through ResNet-152 on ImageNet classification), we could empirically verify that predicted rankings align with expected performance hierarchies. Such validation would confirm that Cross-Select's learned compatibility function respects established model quality gradients rather than introducing spurious ranking artifacts.

Second, replacing the current autoencoder-based model parameter encoder with alternative architectures—including variational autoencoders (VAEs) that capture probabilistic parameter distributions or transformer-based encoders that leverage self-attention over parameter tensors—could reveal whether the bottleneck representation fundamentally limits generalization performance observed in Quadrants II and IV. Comparative ablations across encoder architectures would isolate the contribution of the encoding strategy to overall system performance and potentially identify more effective approaches for translating model parameters into embeddings that generalize to unseen architectures.
