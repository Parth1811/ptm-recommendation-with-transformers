\section{The Model Recommendation Problem}
\label{sec:model-recommendation-problem}

The democratization of pre-trained models has given rise to a paradox of choice: while repositories such as Hugging Face Hub now host over two million models~\cite{liu2025ptmpicker}, practitioners face the daunting challenge of identifying which model best suits their specific application without exhaustive empirical evaluation. This model recommendation problem manifests across three critical dimensions. First, the sheer scale of available models renders manual exploration infeasible, as even domain experts cannot meaningfully assess thousands of candidates varying in architecture, training regime, and task specialization~\cite{meng2023foundation}. Second, the computational cost of fine-tuning or even evaluating each candidate model on target data becomes prohibitive, particularly for resource-constrained practitioners who would benefit most from transfer learning~\cite{tan2024handling}. Third, deployment constraints introduce additional complexity: hardware limitations, latency requirements, and energy budgets—especially critical for edge devices and Internet-of-Things applications—further restrict the feasible model space and necessitate recommendations that account for both task performance and operational constraints~\cite{schwartz2020green,wu2022sustainable}.

The confluence of these challenges motivates the development of automated model recommendation systems that can efficiently predict model suitability without requiring exhaustive fine-tuning experiments~\cite{bytez2025improving,liu2025ptmpicker}. Such systems must balance predictive accuracy with computational efficiency, generalize across heterogeneous tasks and domains, and ideally incorporate deployment-aware considerations to ensure recommended models satisfy both performance and resource constraints. Addressing this problem requires moving beyond traditional model selection heuristics toward learning-based approaches that can capture complex relationships between model characteristics, dataset properties, and task requirements.
